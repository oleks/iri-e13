% vim: set spell:

\section{1}

This exercise was solved exclusively using classic Unix utilities\footnotemark,
specifically \texttt{awk}, \texttt{cat}, \texttt{cut}, \texttt{grep},
\texttt{head} \texttt{sed}, \texttt{uniq}, \texttt{sort}, \texttt{tail} and
\texttt{wc}.  This was done in part to explore the power of these tools, and in
part out of considerations for the reader: it is assumed that the reader, if
nothing else, has had an introduction to a Unix-like operating system. For a
detailed discussion of how the results below were attained, see Appendix
\ref{appendix:1}.

\footnotetext{See also \url{https://en.wikipedia.org/wiki/List_of_Unix_utilities}.}

\begin{itemize}

\item \emph{What is the mean number of queries per user id?}

The number of queries per user is roughly $4.3\pm135.5$, meaning that roughly
68\% of the users perform between $1$ and $140$ queries.

\item \emph{Analyze the variability of query length (i.e., in words or in
characters).}

\item \emph{Is query volume constant throughout the day?}

The query log only covers the morning (or evening) hours, starting at 09:00:00,
and ending at 11:56:18. Within this time frame, the query volume is fairly
constant, as \referToFigure{times} illustrates. The query volume does fall off
after 11:50\footnotemark.

\footnotetext{Perhaps because half of the people go off to lunch \smiley{}.}

\includeFigure[0.6]{times}{Distribution of the query volume over time. The
labels indicate a starting point, with each bar representing a 10 minute time
frame.}

\end{itemize}

