% vim: set spell:

\section{}

This exercise was solved exclusively using classic Unix utilities\footnotemark,
specifically \texttt{awk}, \texttt{cat}, \texttt{cut}, \texttt{grep},
\texttt{head} \texttt{sed}, \texttt{uniq}, \texttt{sort}, \texttt{tail} and
\texttt{wc}.  This was done in part to explore the power of these tools, and in
part out of considerations for the reader: it is assumed that the reader, if
nothing else, has had an introduction to a Unix-like operating system. For a
detailed discussion of how the results below were attained, see Appendix
\ref{appendix:1} (p. \pageref{appendix:1}).

\footnotetext{See also \url{https://en.wikipedia.org/wiki/List_of_Unix_utilities}.}

\begin{itemize}

\item \emph{What is the mean number of queries per user id?}

We assume that a unique hashed user id designates a unique user id.
The number of queries per user is roughly $4.3\pm135.5$, meaning that roughly
68\% of the users perform between $1$ and $140$ queries.

\item \emph{Analyse the variability of query length (i.e., in words or in
characters).}

We extract only the queries from the data file and sort them in lexicographic
order. By looking at the tail of this file, we find that some queries are
written in character-based languages. We chose therefore to analyse the
variability of query length in terms of characters.

A query is roughly $20.5\pm 14.4$ characters long, meaning that roughly 68\% of
the queries have a character length between $6.1$ and $34.9$.

\item \emph{What percentage of queries are mixed case? Upper case? Lower case?}

We assume that this question asks to analyse the casing of all queries, not
just unique queries.

We first normalize the data by removing all non-alphabetic characters from the
queries, and removing those queries which are thus rendered empty. We then
count how many of those queries are all upper-cased, and how many all
lower-cased, the remainder is mixed case. See \referToTable{casing} for
results.

\begin{table}[h!]
\centering
\begin{tabular}{|c|c|c|c|}
\hline
\textbf{Mixed} & \textbf{Upper} & \textbf{Lower} & \textbf{Other} \\
\hline
29.262 \%      & 4.624 \%       & 66.008 \%      & 0.106 \% \\
\hline
\end{tabular}
\caption[]{Percentage of queries using mixed, upper, and lower case. ``Other''
are those queries that contain no alphabetic characters.}
\label{table:casing}
\end{table}

\item \emph{Count the number of questions (look for patterns such as starting
with Wh-words, or ending with a '?'    symbol). What percentage of queries do
questions make up? What is the most common type of question?}

We've compiled a set of interrogative pronouns, and allowed for some variation
in them due to e.g. genitive casing. Looking for these or queries ending in
question mark, we've found that 82045 queries look like questions, or 8.235\%
of the queries.

The most common query seems to be ``Where can I find Math jokes?''. To be
certain to not fall prey to e.g. casing and spacing variations, we've
normalized the data set by lowercasing all the questions, removing spaces, and
all non-alphabetic symbols.

\item \emph{What are the $k$ most common queries issued?}

The dataset was normalized as discussed above, below we present the collection
frequency along with the normalized question.

\begin{lstlisting}
    114 wherecanifindmathjokes
     64 wherecanifindinformationoncaptech
     60 howdoesdryicework
     51 areyougay
     46 whyisdilbertsosexy
     44 whereistheisleofman
     40 howcanjeeveshelpmeshopforholidays
     36 amiinlove
     32 whyistheskyblue
     31 doesbritneyspearshavebreastimplants
     30 isjeevesgay
     29 howoldami
     29 howcanjeeveshelpmeshopfortoys
     28 whattimeisit
     27 whatisthemeaningoflife
     25 whatismyiq
     24 wherecanifindprinterdriversforahewlettpackardlaserjetiiiprinter
     23 whataretheprivateorganizationdoingtohelpalleviatepoverty
     23 howcanjeeveshelpmeshopforgifts
     21 wherecanifindsonglyrics
\end{lstlisting}

\item \emph{What percentage of queries contain stopwords like ``and'', ``the'',
``of'', `in'', ``for''?}

We assume that the intent is for us to use the handed out list of stop words.

For this question (and the next two) we will normalise our queries as follows:
we'll remove all non-alphanumeric characters, turn any sequence of spaces into
exactly one space, and lower case all the queries.

$23.574\%$ of the queries contain stop words.

\item \emph{What are the $k$ most common non-stopwords appearing in queries
that contain the word download?}

Here we first split our queries into words, then remove those words occurring
in the stop words file, sort the remaining words, and count their occurrences.
We present the results below, along with the number of occurrences of that
word:

\begin{lstlisting}
   3441 download
   1110 downloads
    837 free
    394 games
    282 mp3
    256 game
    237 music
    133 downloadable
    109 software
    108 pokemon
     92 windows
     78 christmas
     70 mac
     68 player
     67 pc
     66 internet
     65 demo
     64 funny
     60 files
     55 cd
\end{lstlisting}

\item \emph{What are the $k$ most common non-stopwords appearing in queries?}

We'll proceed as above, except that instead of picking out those queries where
the word ``download'' appears, we'll simply split all the normalized
queries into words. The results are as follows:

\begin{lstlisting}
  20934 free
  13306 pictures
  13117 information
  12382 christmas
  11500 sex
   9369 nude
   8279 new
   7108 buy
   6775 pics
   6708 online
   5670 web
   5501 music
   5232 games
   4687 women
   4272 does
   4177 stories
   4126 site
   4021 cards
   3966 porn
   3850 state
\end{lstlisting}

At this point, we might guess that the data set is from Ask Jeeves from around
Christmas time.

\item \emph{What percentage of queries were asked by only one user?}

We can find this by extracting the hashed user ids from the query log, and
counting those user ids that occur only once. The result is $10.081\%$ of the
queries.

\item \emph{How often is a consecutive query a reformulation of the previous
one? (Not the same query to greater depth.)}

\textbf{Not completed.}

We could solve this by, for every user, for every query, looking multiple
queries ahead, and computing the edit distances\cite[\textsection 3.3.3 (pp.
58--60)]{irbook} to them. Choosing some lower bound, non-zero edit distance,
would help us find some reformulations. As an alternative we could look at the
edit distance in terms of words rather than characters.

\item \emph{What kind of spelling mistakes do users seem to make in general?}

\textbf{Not completed.}

We could find this by looking at how alike words tend to be. First, we'd
normalize the queries by removing some special symbols e.g. commas, and leaving
others, e.g. apostrophes. We then extract the words, and sort them using the
edit distance metric, as above.

This is probably fairly expensive computationally. One alternative would be to
find a list of common spelling mistakes\footnotemark, and look for those
amongst the queries. This however, seems fairly counter-productive as we're
using common spelling mistakes to find common spelling mistakes, yielding
little effective benefit from our work.

\footnotetext{E.g. \url{https://en.wikipedia.org/wiki/Commonly_misspelled_English_words}.}

\item \emph{What percentage of queries contain a person's name?}

\textbf{Not completed.}

Recognising names is a fairly complicated task. First we'd normalize the
dataset by deleting all digits, genitive casing, special symbols, etc. Then
we'd match every word in a query against a list of names.

A different approach would be to match every word against a specially crafted
English dictionary, that did not contain names. Whenever we find a word that is
not in this dictionary, it is likely to be a name. This strategy is perhaps
safer as names, other than being quite numerous, also may have a lot of
variations in their transliterations to English. Besides, users tend to make
spelling mistakes.

\item \emph{How often do URLs appear in queries?}

Looking for queries that contain a contiguous sequence of non-space, and non-at
(@) characters, followed by a dot and a sequence of alphabetic characters of
length at least two (the top-level domain), reveals that 4.644\% of the queries
look like they contain a URL.

This strategy is fairly na\"ive, as this also matches e.g. searches for
\texttt{.exe} files.

\item \emph{Is it likely that this web query log puts anyone's privacy at
risk?}

The web query log itself does not put anyone's privacy at risk, so long as
users are not personally identifiable. Processing some of these queries might.
Some queries seek to infringe the privacy of others, while others yet might
seek to attack the search engine itself, and thereby compromise private data on
the search engine's servers.

\item \emph{Can you find addresses, phone numbers, or other identifiers in the
log file?}

Having little knowledge of the geographical origin of the queries, this is
quite a challenging task. For instance, not all addresses contain numbers, and
not all phone numbers are formatted alike.

One straight forward approach to looking for phone numbers is to looks for
queries with long sequence of digits, possibly separated by spaces or dashes.
Having removed the queries that look like having years and year ranges, this
approach still does not yield very satisfactory results.

Addresses are even more complex in this sense. Addresses don't necessarily have
any keys or special separators. One possible strategy would be to find queries
with a lot of separators (e.g. comma), having digits, and being mixed case.
This was not attempted.

\item \emph{Is query volume constant throughout the day?}

The query log only covers the morning (or evening) hours, starting at 09:00:00,
and ending at 11:56:18. Within this time frame, the query volume is fairly
constant, but slightly larger in the first half hour, as \referToFigure{times}
illustrates. The query volume does fall off after 11:50\footnotemark, but this
may be an outlier.

\footnotetext{Perhaps because half of the monkeys have gone off to lunch
\smiley{}.}

\includeFigure[0.6]{times}{Distribution of the query volume over time. The
labels indicate a starting point, with each bar representing a 10 minute time
frame.}

\end{itemize}

