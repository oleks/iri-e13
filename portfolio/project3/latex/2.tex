% vim: set spell:

\section{Recommendations for Amazon}

\begin{enumerate}[a.]

\item

\begin{enumerate}[1.]

\item Amazon recommendations can be classified as a form of query term
suggestions \cite[\textsection\ 1.5.4]{hearst}. The information need however is
less explicit. When a user follows recommendations, it is only sometimes to
fulfill the original information need. Recommendations offer similar material
that may also be of interest. They fulfill an information need which the user
was not necessarily consciously aware of.

\item Amazon reviews do not immediately fit any of the recommendations for
search user interface design presented in the literature. One way that
Amazon.com could improve upon these is to allow users to add tags to products,
so that richer networks of categories of products can emerge. This would also
provide information for the search engine, which would otherwise require more
complicated processing of user reviews.

\end{enumerate}

\item

\begin{enumerate}[1.]

\item Let a user pick a default shipping address. Analyze this address before
presenting an item as either a search result and presenting a buy or a wish
list button. If the item cannot be delivered to the default shipping address,
do not show it in search results (unless perhaps the users explicitly asks for
this), and instead of a buy or add to wish list button, inform the user of the
fact that this item cannot be delivered.

This way, a user is never let astray believing that they can obtain an item,
when they indeed cannot.

\item Consolidate all the web sites into one. In search results, present only
those items that are deliverable to the users default address, or approximate
location based on IP if no address has yet been specified. It may be possible
to order the same item from different locations (e.g. United States, United
Kingdom, Germany). Pick the one that leads to cheapest delivery price and
lowest delivery time. Give the user choose another country.

This way, a user is never let astray believing that they can obtain an item,
when they indeed cannot.

This option is extremely costly as it demands a rework of the entire Amazon.com
brand. A simpler option would be to give AmazonGlobal a more clear presence on
the internet. This only slightly reduces the risk of letting users astray as
discussed above.

\item Show the total amount to be paid when asking for credit card information.
Although the home currency of the buyer may not revealed yet, the buyer is full
aware of the currency used to price the item they have just proceeded to
attempt to buy.

\item Make the distinction more clear, perhaps with the use of background
colours, whether a book is physical or an Amazon Kindle book. Do this both in
search results and the product pages of the books. The distinction is important
enough to deserve this treatment, as a reader without an Amazon Kindle reader
cannot read an Amazon Kindle book, and would feel mislead having purchased such
a book.

\item Put the shipping rates and times together, or at least have them no more
than back-click and hyperlink away from one another. This way the users will
find delivery information more quickly, and proceed with (or quit) shopping.

\item Do not let items that cannot be delivered to the users default address be
added to their wish list. This way, the users won't build up wish lists of
items they (and most likely their family and friends) cannot buy.

\item Don't save items removed from the shopping cart for later purchases,
especially when these items could not be delivered to the address specified by
a user.

\item Let the pictorial representation adhere to the order of elements in the
search results when sorted by rating, or vice versa. This will avoid users
getting confused when seeming lower rated items precede higher rated items when
sorted by nondecreasing rating.

\end{enumerate}

\item The below are inspired by \cite[\textsection\ 1]{hearst}.

\begin{itemize}

\item Make it easier for users to recover from errors and avoid them
altogether. For instance, make sure that it's not possible to place an item
into the shopping cart which cannot be shipped to their address. Removing an
item from the shopping cart is a laborious process, consider a step-counter
alongside the text box.

\item Inform the users more clearly of what is going on. The user tests showed
problems with seeing the notifications that the items could not be delivered to
the particular address, and switching to Kindle versions of books rather than
physical. See also problem 7.

\item Suggest that the user makes an explicit choice when searching for book
titles, whether they would like to see results for Amazon Kindle books or
physical books.

\end{itemize}

\end{enumerate}

