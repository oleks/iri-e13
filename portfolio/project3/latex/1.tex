% vim: set spell:

\section{User test of Amazon}

Amazon.com, Inc. is a global online retailer. \texttt{amazon.com} is the main
subsidiary of Amazon.com, Inc. headquartered in the United States. The main
target audience of \texttt{amazon.com} is hereby the United States residents.
\texttt{amazon.com} also offers a wide range of international delivery options,
broadening the target audience to most of the rest of the world, with but a
few exceptions.

This makes it hard to address a representative subset of the
\texttt{amazon.com} target audience within the limits of this assignment. We
will constrain ourselves to Danish residents.

For Denmark, there are good delivery options\footnotemark, and the Danish
residents have a high level of internet literacy\cite{bebrit02}. In fact, 72\%
of Danish residents bought items online in 2013\cite{bebrit07}. 42\% of the
residents bought items from online retailers outside of Denmark but within the
EU, and 17\% of the residents bought items from outside the EU\cite{bebrit11}.

\footnotetext{See
\url{http://www.amazon.com/gp/help/customer/display.html?nodeId=596194} for
shipping rates, and
\url{http://www.amazon.com/gp/help/customer/display.html?nodeId=201118410} for
shipping times.}

Amazon.com has multiple internationally-focused web sites, but no online
presence in the Danish top-level-domain. It varies which subsidiary Danish
residents prefer. According to \cite[FDIM Top 1000]{brug-2012},
\texttt{amazon.com} had about 219 thousand unique visitors in the second
quarter of 2011, while \texttt{amazon.co.uk} had 188, and \texttt{amazon.de}
had 34 thousand. Alexa (a daughter company of Amazon.com) in January 2014
rates\footnotemark~\texttt{amazon.com} 18th most popular web site in Denmark,
\texttt{amazon.co.uk} 28th, and \texttt{amazon.de} 288th most popular.

\footnotetext{See also \url{http://www.alexa.com/siteinfo/amazon.com}.}

Amazon.com consolidates all the accounts on its subsidiaries, meaning that a
user account created on e.g. \texttt{amazon.co.uk}, is also available on
\texttt{amazon.com}. Despite this, little data is shared among the
subsidiaries, e.g. wish lists differ.

Concluding from the discussion above, it is hard to find test users among the
Danish residents who have never used an Amazon.com subsidiary, as suggested by
\cite{molich}. We choose to pick test users who already have Amazon.com
accounts and have perhaps tried purchasing at least one item from one of its
subsidiaries. We will have users set up with a dummy user account.

% As we are the target audience of 

% The experimenter in our case is also the user representative. 

% Use a dictionary, formulate in a way as a Danish-English dictionary would translate it.

% Place an order for a Kindle.

% Test location - easy, use your own computer in private.

% Pilot test.

% Debrief.

% Designed for 30 minutes, since no one should really spend longer on Amazon.

% Hard to find users that have not used the application before.

% Evaluate the traditional way as well?

% Would be nice to do eye tracking - measure what they actually see.

% Track screen - measure where they actually click.

% The users already have Amazon accounts.
% - The delivery fees might have changed, figure out the delivery options.
% - The products you seek are likely to not be available outside the US.
%   - e.g. special books
% - Kindle - how can I get a Kindle in Denmark?

% Easy to avoid bias - we didn't design the application so have little
% attachment to it.

% Instead of picking the users, we may pick the tasks, due to Amazon.com's
% diversity in choice of products.

% Easy to design tasks that reflect what the users would actually like to do
% with the system.

\subsection{Preparation}

We will be conducting a thinking aloud test, half-way between a formal and
discount usability test. We'll attempt to stay as formal as feasible for a
group of one within the time frame of this assignment.

\subsubsection{Test environment}

The test is to be conducted on a test machine so that recording can easily take
place. In our tests we will use the Chromium web browser, and record the
session with \texttt{simplescreenrecorder}\footnotemark. We would also have
liked to perform eye-tracking, but unfortunately, due to software limitations,
this was not feasible.

\footnotetext{See also \url{http://www.maartenbaert.be/simplescreenrecorder/}.}

For the purposes of this experiment a test user has been set up on
\texttt{amazon.com}. The test user was used to run the pilot test, leaving some
user data behind on \texttt{amazon.com}, including a delivery address. This has
an effect on what suggestions \texttt{amazon.com} comes with.

Before the test user arrives, the following should be completed:

\begin{enumerate}

\item Print the instructions and task set for the test users.

\item Set up a machine with a comfortable seating position.

\item Open the Chromium browser.

\item Log in to \texttt{amazon.com} with the test user.

% amazon.test.42@gmail.com

\item Return to the main \texttt{amazon.com} page, if not already there.

\item Initialize a screen recording session of the browser window.

\end{enumerate}

\subsubsection{Test users}

We would like to evaluate the Amazon.com search user interfaces. As such, we
will avoid making any purchases in our tests. The test users should be informed
of this, as some users do not appreciate making purchases on foreign computers.
The test users are told to imagine that they are using a friend's computer.

The user should in particular be informed that their session will be recorded
and made publicly available. They should not enter any visible confidential
information in the course of the test. We also inform the users that they will
be using a dummy account which has entered a single delivery address, and was
used for a pilot test.

Additionally, we ask the test users the following questions before the test in
order to establish a context:

\begin{enumerate}

\item Have you used \texttt{amazon.com}, \texttt{amazon.co.uk}, or
\texttt{amazon.de} before?

\item Have you used the wish lists functionality before?

\item Do you own an Amazon Kindle? If so, did you purchase it from
\texttt{amazon.com}, \texttt{amazon.co.uk}, or \texttt{amazon.de}?

\end{enumerate}

\subsubsection{Tasks}

The user is asked to proceed with the following tasks on their own, with only
moderate guidance by the experimenter in times of despair. The tasks below are
therefore directed towards the test user. Note that language is also carefully
chosen to be European, i.e. British in nature, e.g. we use the term
``delivery'' rather than ``shipping''.

\begin{enumerate}

\item Imagine you are at a friend's house. While your friend is preparing the
last of your evening meal you get to use the laptop in the dining room. You
have already logged into \texttt{amazon.com} with the account you created a
couple weeks ago. The experimenter is your friend, and an experienced
\texttt{amazon.com} user.

\item In case you need, the Test User password: MyTh1sIsP3swrd.

\item Find out what the delivery rates and times are for delivering books to
Denmark.

\item You've spotted this nice book at your friend's house, Zen and the Art of
Motorcycle Maintenance. It's not worth the trouble buying right now, but you
would consider it a nice gift. Find the book and add it to your wish list.

\item You'd also like a good, introductory book on the subject of cryptography.
\texttt{amazon.com} has ratings and reviews of the books in store. Find one or
two good books and add them to your wish list.

\item You've spotted a neat new Kindle at your friend's house. You would like
to see what different models there are. Find a model that you like and that you
can get delivered to Denmark. How much would it cost, including delivery?

\end{enumerate}

\subsubsection{Topics for debriefing}

We propose the following topics for the debriefing after the test, although
they are merely for guidance. Problems that arise during the test in general,
are ideal topics for debriefing.

\begin{enumerate}

\item Do you find that \texttt{amazon.com} is good at informing you of which
products you can and cannot get delivered to Denmark? How can they improve
this?

\item Do you find that the \texttt{amazon.com} review and rating system is
reliable?

\end{enumerate}

\subsection{Tests \& Results}

The recordings of the user tests \#1 and \#2 are made available on YouTube for
the purposes of verification of this analysis. The links are, respectively:

\begin{enumerate}

\item \url{http://www.youtube.com/watch?v=HO9PLn69TkM}

\item \url{http://www.youtube.com/watch?v=dlqZ_A3onSg}

\end{enumerate}

We proceed with the analysis below.

\subsubsection{Time}

The tests were designed to take roughly half an hour, including debriefing,
because it is unlikely that users in a natural setting would spend time longer
than this on \texttt{amazon.com}. In practice, the users took roughly 20
minutes to complete the entire test.

\subsubsection{Good features}

\begin{enumerate}

\item The test users found the suggestions given by \texttt{amazon.com} when
looking at books, adding to wish lists, and adding items to the shopping cart,
to be useful and spent time browsing this way.

\item The \texttt{amazon.com} reviews can give some context to books you're
searching for if you are unfamiliar with the subject.

\end{enumerate}

\subsubsection{Problems}

First, we'll list the problems that can be identified as User Interface
Disasters (UID), as they were observed in both user tests.

\begin{enumerate}

\item \texttt{amazon.com} allows to add items to a shopping cart which cannot
be delivered to an address specified in the user profile.

Both users added a Kindle that they could not get delivered to their address in
Denmark. The users had to explicitly remove the items from the shopping cart
and find alternatives. User \#2 repeated this via \texttt{amazon.co.uk} after
being redirected there from \texttt{amazon.com}, only to put another Kindle
into the shopping cart which cannot be delivered to Denmark. The said user was
in the end unable to find an Amazon Kindle that could be.

The delivery address section in the user profile allows one to specify multiple
addresses in completely different geographical regions. It is presumably a
concious design decision not to consider the user's delivery address profile
when presenting the shopping items on \texttt{amazon.com}. Browsing from
Denmark, having just one delivery address, both users found this confusing.

We would classify this problem as critical as it prohibits users from buying an
Amazon Kindle for mere usability reasons.  Indeed, user \#2 was unable to place
a valid order. In case of a targeted purchase, when a user clicks to add an
item to the cart, this is conceptually equivalent to them taking out a wallet
at a store. The store then rejects their purchase saying that the item cannot
be bought.

\item It is only when the user enters the special Amazon.com (intl.), aka.
AmazonGlobal, which has no well-designated URL, that a user can order an Amazon
Kindle that can be delivered to Denmark.

We would classify this problem as critical as it prohibits users from buying
via Amazon.com for mere usability reasons.

\item \texttt{amazon.com} demands credit card data before showing the final sum
of money to be paid. Neither user could present the final price of an Amazon
Kindle, including VAT and shipping before having to enter credit card
information.

We would classify this problem as serious as this is where the customer
physically get their wallets out, but do not feel safe as they do not know how
much they would have to put down.

\end{enumerate}

Now we turn to milder problems that were only identified by one of the user
tests:

\begin{enumerate}

\setcounter{enumi}{3}

\item There is little difference between how Books and Kindle Books are
presented, so that the user sometimes finds these to be inconsistencies in the
design rather a change in product category. This was identified by user \#2.

We would classify this problem as serious, as it may lead users to place
unwanted purchases, or wish for unwanted items.

\item The shipping rates and times information, though connected in hyperlink
titles, are not connected in proximity. It is only the many ways of finding
both that lets some users find this information quickly. Test user \#2
struggled with this.

We would classify this problem as cosmetic. Shipping information is always
present at the time of checkout, and there are many ways of obtaining this
information in general.

\item It is unclear whether the items in the user's wish list can at all be
delivered to the address(es) in their profile. This was identified by user \#2
during debriefing.

We would classify this problem as serious, as this may lead users to not use
the wish list feature which is otherwise an excellent bookmarking facility for
later purchases.

\end{enumerate}
