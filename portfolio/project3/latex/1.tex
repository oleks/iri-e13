% vim: set spell:

\section{User test of Amazon}

Amazon.com, Inc. is a global online retailer. \texttt{amazon.com} is the main
subsidiary of Amazon.com, Inc. headquartered in the United States. The main
target audience of \texttt{amazon.com} is hereby the United States residents.
\texttt{amazon.com} also offers a wide range of international delivery options,
broadening the target audience to most of the rest of the world, with but a
few exceptions.

This makes it hard to address a representative subset of the
\texttt{amazon.com} target audience within the limits of this assignment. We
will constrain ourselves to Danish residents.

For Denmark, there are good delivery options\footnotemark, and the Danish
residents have a high level of internet literacy\cite{bebrit02}. In fact, 72\%
of Danish residents bought items online in 2013\cite{bebrit07}. 42\% of the
residents bought items from online retailers outside of Denmark but within the
EU, and 17\% of the residents bought items from outside the EU\cite{bebrit11}.

\footnotetext{See
\url{http://www.amazon.com/gp/help/customer/display.html?nodeId=596194} for
shipping rates, and
\url{http://www.amazon.com/gp/help/customer/display.html?nodeId=201118410} for
shipping times.}

Amazon.com has multiple internationally-focused web sites, but no online
presence in the Danish top-level-domain. It varies which subsidiary Danish
residents prefer. According to \cite[FDIM Top 1000]{brug-2012},
\texttt{amazon.com} had about 219 thousand unique visitors in the second
quarter of 2011, while \texttt{amazon.co.uk} had 188, and \texttt{amazon.de}
had 34 thousand. Alexa (a daughter company of Amazon.com) in January 2014
rates\footnotemark~\texttt{amazon.com} 18th most popular web site in Denmark,
\texttt{amazon.co.uk} 28th, and \texttt{amazon.de} 288th most popular.

\footnotetext{See also \url{http://www.alexa.com/siteinfo/amazon.com}.}

Amazon.com consolidates all the accounts on its subsidiaries, meaning that a
user account created on e.g. \texttt{amazon.co.uk}, is also available on
\texttt{amazon.com}. Despite this, little data is shared among the
subsidiaries, e.g. wish lists differ.

Concluding from the discussion above, it is hard to find test users among the
Danish residents who have never used an Amazon.com subsidiary, as suggested by
\cite{molich}. We choose to pick test users who already have Amazon.com
accounts and have tried purchasing at least one item from one of its
subsidiaries. We will instead have a test set up procedure where the test users
are asked to login with their account.

% As we are the target audience of 

% The experimenter in our case is also the user representative. 

% Use a dictionary, formulate in a way as a Danish-English dictionary would translate it.

% Place an order for a Kindle.

% Test location - easy, use your own computer in private.

% Pilot test.

% Debrief.

% Designed for 30 minutes, since no one should really spend longer on Amazon.

% Hard to find users that have not used the application before.

% Evaluate the traditional way as well?

% Would be nice to do eye tracking - measure what they actually see.

% Track screen - measure where they actually click.

% The users already have Amazon accounts.
% - The delivery fees might have changed, figure out the delivery options.
% - The products you seek are likely to not be available outside the US.
%   - e.g. special books
% - Kindle - how can I get a Kindle in Denmark?

% Easy to avoid bias - we didn't design the application so have little
% attachment to it.

% Instead of picking the users, we may pick the tasks, due to Amazon.com's
% diversity in choice of products.

% Easy to design tasks that reflect what the users would actually like to do
% with the system.

\subsection{Set up}

The test is to be conducted on a set up machine so that recording can easily
take place. In our tests we will use the Chomium web browser. We will record
the browser window with \texttt{simplescreenrecorder}\footnotemark. Ideally, we
would also like to perform eye-tracking, but unfortunately, due to software
problems, we will not do this in our tests.

\footnotetext{See also \url{http://www.maartenbaert.be/simplescreenrecorder/}.}

We would like to evaluate the Amazon.com search user interfaces. As such, we
will avoid making any purchases in our tests. The test users should be informed
of this, as some users do not appreciate making purchases on foreign computers.

\begin{enumerate}

\item Have you used \texttt{amazon.com}, \texttt{amazon.co.uk}, or
\texttt{amazon.de} before?

\end{enumerate}

The user is guided by the experimenter through the set up procedure. The tasks
below are therefore directed towards the experimenter.

\begin{enumerate}

\item Ask the user for their consent to record the desktop session for the
purposes of review.

\item Inform the user that it is \texttt{amazon.com} that is being tested and
not them.

\item Inform the user that their Amazon.com account, even if created on e.g.
\texttt{amazon.co.uk} also exists on \texttt{amazon.com}.

\item Ask the user to go to \texttt{amazon.com} and login with their account.

\item Ask the user to go back to the main page of \texttt{amazon.com}, if not
already there.

\end{enumerate}

% Would you buy from someone else's computer?
% No actual purchase is being made.

\subsection{Tasks}

The user is asked to proceed with the following tasks on their own, with only
moderate guidance by the experimenter in times of despair. The tasks below are
therefore directed towards the test user.

\begin{enumerate}

\item Find out what the delivery rates and times are for delivering books to
Denmark.

\item 

\end{enumerate}
