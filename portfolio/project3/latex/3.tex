% vim: set spell:

\section{Information seeking on Amazon}

\begin{enumerate}[a.]

\item

\begin{itemize}

\item As an online retailer, \texttt{amazon.com} possesses certain information
regarding the process of retail itself. A user seeking to purchase items might
be interested in this information, e.g. for the purposes of ensuring that the
items they purchase can indeed be delivered to them. These are instances of the
standard information seeking model, which have the potential to grow into
instances of the sensemaking model. Indeed, once under the Help section,
\texttt{amazon.com} provides an additional search box for searching through
this section.

Task 3 was an example of this.

Amazon is fairly good at this by providing a wealth of pathways to the
information, but irritates the user as seemingly related information is
sometimes in low proximity of one another, such as the delivery times and
prices.

\item Amazon has a variety of products for purchase. A user may come to the web
site with a specific book, or other product in mind. Such scenarios also adhere
to the standard information seeking model, but instead have the potential to
grow into instances of the dynamic(berry-picking) model. Indeed, Amazon has a
wealth of instances of product recommendations, and users often diverge on
these paths in the process.

Task 4 is an example of this.

Amazon is fairly good at this with the exception that they do not make the
distinction between physical and Kindle books clear enough for the user to not
be let astray.

\item Amazon has product ratings and reviews. In case of books, the reviews may
be used for the purposes of sensemaking about a particular topic, and the
ratings may be used to figure out standard literature on a particular topic.

Task 5 is an example of this.

Amazon is fairly good at this, but as user test \#2 showed, a user might be a
little confused by the way that \texttt{amazon.com} sorts user ratings. What
Amazon is doing bad however is that there is no seeming way of building up
trust to reviewers. Users would like to know to what extent what they are
reading is an expert review, as was discovered during the debriefing of test
\#1.

\item Amazon makes their own products, among them the Amazon Kindle. A user
might be looking for an item in a particular category but has not completely
narrowed down their information need. For instance, what types of Kindles there
are.

Task 6 is an example of this.

Amazon is okay at presenting their product line, but not very good at
presenting it in a manner so that a Danish resident can purchase a product
right away.

\end{itemize}

Looking at these formal models relieves us of the bias induced by e.g. the
financial incentive, i.e. that users should buy Amazons products. It allows us
to focus on the satisfaction of a fundamental user need, the information need.

\item

A search box for searching through the store is always available. This search
box does not have greyed out text to indicate what can be searched for, but its
breadth and location indicate its globality\cite[\textsection\ 1.7,
1.10]{hearst}. On the left-hand side of the search box, the user can pick and
choose certain categories among which to search. These categories are also
available on the left hand side in more fine-grained form. Categories also show
up in grayed out text next to the search suggestions when a user is typing a
query. All this sums to solid attempts to make the user narrow down their
search by picking categories\cite[\textsection\ 8.2]{hearst}. Despite their
efforts, our user tests show that users are still led astray.

When entering the help section, an additional help search box appears. This
search box, also has a greyed out text indicating what can be searched for
using this search box.

For a particular search result, \texttt{amazon.com} also makes a good job of
showing all the different options available for that product. For instance, the
different pricing options, and for books, cover options, Kindle and Audio CD
versions, etc.

\texttt{amazon.com} also has a variety of ways to sort the search
results\cite[\textsection\ 8.3]{hearst}. Although some of these seem a bit
confusing to users, they do help in instances of sensemaking about particular
topics, where standard literature is sought.

Evidently, \texttt{amazon.com} also practices faceted navigation
\cite[\textsection\ 8.6]{hearst} as users ideally first pick a subsidiary
focused on their geographical region, then a product category, and once they've
found a product, a product type, e.g . a physical book vs. a Kindle book. The
order of these choices, however, tends to get mixed up as the user progresses
with their information need, as our user tests have shown.

\end{enumerate}
